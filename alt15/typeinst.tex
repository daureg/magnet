\documentclass[runningheads,a4paper]{llncs}
\usepackage[utf8]{inputenc}
\usepackage{amssymb}
\setcounter{tocdepth}{3}
\usepackage{graphicx}

\usepackage{url}
\newcommand{\keywords}[1]{\par\addvspace\baselineskip
\noindent\keywordname\enspace\ignorespaces#1}

\begin{document}

\mainmatter
\title{Low stretch spanners in signed graphs for active links sign prediction}
\titlerunning{Lecture Notes in Computer Science: Authors' Instructions}
\author{Géraud Le Falher \and Marc Tommasi \and Fabio Vitale}
\authorrunning{Géraud Le Falher, Marc Tommasi and Fabio Vitale}
\institute{%
	Inria Lille -- MAGNET, CRIStAL, UMR 9189, Villeneuve d'Ascq, France\\
	\url{{geraud.le-falher,marc.tommasi,fabio.vitale}@inria.fr}}
\toctitle{Lecture Notes in Computer Science}
\tocauthor{Authors' Instructions}
\maketitle


\begin{abstract}
The abstract should summarize the contents of the paper and should
contain at least 70 and at most 150 words. It should be written using the
\emph{abstract} environment.
\keywords{active learning; signed graph; low-stretch; metric embedding}
\end{abstract}


\section{Introduction}

\section{Methods}

\section{Analysis}

\section{Experiments}

Why we choose $z=15$ for Asym: $i)$ one of the best in \cite[Fig.
11]{Kunegis2009}, $ii)$ perform well on real dataset in
\cite[Fig.3]{Cesa-Bianchi2012a}, $iii)$ good in our initial testing
(\texttt{20150401\_wed\_spectral.ipynb}

\section{Related work}

\section{Conclusion}

\bibliographystyle{splncs03}
\bibliography{../../../biblio/magnet.bib}

\end{document}
