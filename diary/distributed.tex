In the 90s, \textcite{Valiant1990} defined a new model of parallel computation
called bulk-synchronous parallel.

\begin{quotation}
	
Valiant observes that parallel processing has not made the expected progress in
displacing sequential processing in computationally intensive domains. He
attributes this failure to the lack of an appropriate bridging model for
parallel computation, analogous to the von Neumann model for sequential
computation. He then proposes the bulk-synchronous parallel (BSP) model as a
candidate bridging model. The key claim concerning BSP is that it provides a
fixed intermediate-level model through which high-level programs can be
efficiently mapped to diverse machine architectures. The paper supports the
claim with evidence of varying sorts: certain high-level constructs (such as
memory hashing) and algorithms are efficiently implemented in BSP, and BSP is
implemented in two low-level models. Most of the arguments are only sketched,
with details left to the references. The BSP model consists of (1) components
(each performing processing or memory functions or both), (2) a point-to-point
routing mechanism for delivering messages between components, and (3)
facilities for periodic synchronization of components. The efficiency of BSP as
a bridging model is improved by parallel slackness in programs; that is, the
number of virtual processes must be sufficiently larger than the number of
processors. Another parameter important to BSP efficiency is the ratio $g$ of
total local computation rate to total router delivery rate. Valiant states that
existing machines have $g$-values higher than ideal for BSP and investment in
communication hardware to reduce $g$ (and keep it small as machine size
increases) will result in more programmable machines. The paper is a brief,
well-written introduction to a research topic. Whether a bridging model is
needed for practical progress in parallel processing, or whether BSP is a
suitable bridging model, remains unclear. The author does not discuss evidence
against the existence of bridging models; for instance, parallel and
distributed processing form a continuum with vastly different properties from
one end to the other. Finally, one might consider the analogous failure of
nuclear fusion reactors to replace fission reactors--is something lacking in our
understanding of fusion, or is fusion simply a more difficult engineering
problem than was first expected?   

Reviewer: Ronald J. Watro 
\end{quotation}

Closer to us, the increasing size of available data has pushed companies to
innovate and Google introduce the Pregel model \autocite{Pregel10}.

While Pregel is not publicly available, several open implementation have been
described, with various modifications

\begin{itemize}
	\item Apache
		Giraph\footnote{\href{http://giraph.apache.org/}{http://giraph.apache.org/}}
	\item Mizan\footnote{Actually, I can't find its implementation} \autocite{Khayyat2013}
	\item
		GPS\footnote{\href{http://infolab.stanford.edu/gps/}{http://infolab.stanford.edu/gps/}}
		\autocite{Salihoglu2013}
\end{itemize}

\Textcite{Han2014} conducted an experimental comparaison of their performance and
usability\footnote{Along with GraphLab} on four tasks: PageRank, Single Source
Shortest Paths, Weakly Connected Components and Minimum Spanning Tree.

Outside the Pregel family (\emph{is it?}), we can cite:

\begin{itemize}
	\item
		GraphLab\footnote{\href{http://graphlab.org/projects/source.html}{http://graphlab.org/projects}}%
		\autocites{Low2010}{Gonzalez2012}{Low2012}{GraphLab13Low} and
		GraphChi\footnote{\href{http://graphlab.org/projects/graphchi.html}{http://graphlab.org/projects/graphchi.html}}
		\autocite{Kyrola2012}.
	\item
		Spark \autocite{Zaharia2010} and its GraphX
		extension\footnote{\href{https://spark.apache.org/graphx}{https://spark.apache.org/graphx}}
	\item FlashGraph\footnote{\href{http://www.cs.jhu.edu/~zhengda/\#FlashGraph}%
			{\url{http://www.cs.jhu.edu/~zhengda}}} \autocite{FlashGraph14}:
		using SSD disks instead of memory to store large graph without
		compromising too much performance.
	\item Using GPU like MapGraph\footnote{\href{http://mapgraph.io/}{http://mapgraph.io/}}
		\autocite{Fu2014},
		\textsc{Totem}\footnote{\href{http://netsyslab.ece.ubc.ca/wiki/index.php/Totem}%
			{http://netsyslab.ece.ubc.ca/wiki/index.php/Totem}}
		\autocite{Gharaibeh2013} or
		Gunrock\footnote{\href{http://gunrock.github.io/gunrock/}%
			{http://gunrock.github.io/gunrock/}} \autocite{Gunrock15}
	\item Asynchronous \autocite{Wang2013}
	\item TinkerPop \footnote{\href{http://www.tinkerpop.com/docs/3.0.0.M5/}%
			{http://www.tinkerpop.com/docs/3.0.0.M5/}}
	\item Focusing not on vertex but on neighborhood allow some algorithms to
		smaller memory footprint andn message overhead than approaches above,
		for instance \textsc{NScale} \autocite{Quamar2014}
\end{itemize}

As hinted by a paper submitted to
HotOS15\footnote{\href{http://www.frankmcsherry.org/graph/scalability/cost/2015/01/15/COST.html}%
	{http://www.frankmcsherry.org/graph/scalability/cost/2015/01/15/COST.html}}
and another
blogpost\footnote{\href{http://aadrake.com/command-line-tools-can-be-235x-faster-than-your-hadoop-cluster.html}%
	{http://aadrake.com/command-line-tools-can-be-235x-faster-than-your-hadoop-cluster.html}},
one must make sure that graph's size require distributed computation, as it
comes with a non negligible overhead. Otherwise, it possible to use single
machine, parallelized algorithms.
Ligra\footnote{\href{https://github.com/jshun/ligra}%
	{https://github.com/jshun/ligra}} \autocite{Ligra13} (and its compressed
version \autocite{Ligra15}) successfully illustrates this approach.

Then talk about related problem like edges partition \autocite{Bourse2014a} or
algorithms optimization \autocites{Salihoglu14}{Salihoglu2014b} (for instance
using algebrisation \autocite{Kaski2014}) and complexity \autocite{Klauck2013}.

Also show some recent applications, for instance in \pcc{}
\autocites{Bonchi2012}{Chierichetti2014}.
\autocite{Quick2012}
