\documentclass[a4paper,final,notitlepage,11pt,svgnames]{article}
\usepackage[english]{babel}
\usepackage[utf8]{inputenc}
\usepackage{amsmath}
\usepackage{amssymb}
\usepackage{microtype}
\usepackage{csquotes}
% \usepackage{fullpage}
\usepackage{charter}
\usepackage{xcolor}
\usepackage{enumitem}
\usepackage{setspace}
\usepackage{booktabs}
\usepackage{caption}
\usepackage{subcaption}
\usepackage[np,autolanguage]{numprint}
\usepackage[final]{pdfpages}
\usepackage[citestyle=numeric-comp,bibstyle=ieee,isbn=false,maxnames=1,minnames=1,sorting=none,backend=biber,defernumbers=true]{biblatex}
\AtEveryBibitem{
   \clearfield{arxivId}
   % \clearfield{booktitle}
   % \clearfield{doi}
   \clearfield{eprint}
   \clearfield{eventdate}
   \clearfield{isbn}
   \clearfield{issn}
   % \clearfield{journaltitle}
   \clearfield{month}
   % \clearfield{number}
   % \clearfield{pages}
   \clearfield{series}
   % \clearfield{url}
   \clearfield{urldate}
   \clearfield{venue}
   % \clearfield{volume}
   \clearlist{location} % alias to field 'address'
   \clearlist{publisher}
   \clearname{editor}
}
\usepackage{hyperref}
\hypersetup{%
    % draft,    % = no hyperlinking at all (useful in b/w printouts)
    colorlinks=true, linktocpage=true, pdfstartpage=3, pdfstartview=FitV,%
    % uncomment the following line if you want to have black links (e.g., for printing)
    %colorlinks=false, linktocpage=false, pdfborder={0 0 0}, pdfstartpage=3, pdfstartview=FitV,%
    breaklinks=true, pdfpagemode=UseNone, pageanchor=true, pdfpagemode=UseOutlines,%
    plainpages=false, bookmarksnumbered, bookmarksopen=true, bookmarksopenlevel=1,%
    hypertexnames=true, pdfhighlight=/O,%nesting=true,%frenchlinks,%
    urlcolor=Chocolate, linkcolor=DodgerBlue, citecolor=LimeGreen, %pagecolor=RoyalBlue,%
}
\begin{filecontents}{jobname.bib}
@article{WTA13,
  author  = {Nicol{\`o} Cesa-Bianchi and Claudio Gentile and Fabio Vitale and Giovanni Zappella},
  title   = {Random Spanning Trees and the Prediction of Weighted Graphs},
  journal = {Journal of Machine Learning Research},
  year    = {2013},
  volume  = {14},
  pages   = {1251-1284},
  url     = {http://jmlr.org/papers/v14/cesa-bianchi13a.html}
}
@article{ImdbData07,
  author  = {Sofus A. Macskassy and Foster Provost},
  title   = {Classification in Networked Data: A Toolkit and a Univariate Case Study},
  journal = {Journal of Machine Learning Research},
  year    = {2007},
  volume  = {8},
  pages   = {935-983},
  url     = {http://www.jmlr.org/papers/v8/macskassy07a.html}
}

\end{filecontents}
\addbibresource{jobname.bib}
\usepackage[hmargin=2.5cm,vmargin=1.5cm]{geometry}

\title{\rta{} first experiments}
\author{Géraud Le Falher}
% \date{}

\newcommand{\rta}{\textsc{RTA}}
\newcommand{\wta}{\textsc{WTA}}
\newcommand{\shazoo}{\textsc{Shazoo}}
\newcommand{\lprop}{\textsc{Lab. Prop}}
\newcommand{\cora}{\textsc{Cora}}
\newcommand{\citeseer}{\textsc{Citeseer}}
\newcommand{\pubmed}{\textsc{Pubmed}}
\newcommand{\usps}{\textsc{USPS-10}}
\newcommand{\rcv}{\textsc{RCV1-10}}
\newcommand{\imdb}{\textsc{IMDB}}

\begin{document}
\maketitle

\section*{Datasets}
\label{sec:Datasets}

The first 3 datasets under consideration are citations
networks\footnote{Available at
\url{http://linqs.cs.umd.edu/projects/projects/lbc/}} (with directions
removed).  Nodes are papers, which belong to a few categories (Agents, AI, DB,
IR, ML and HCI for \citeseer{}; Case Based, Genetic Algorithms, Neural
Networks, Probabilistic Methods, Reinforcement Learning, Rule Learning and
Theory for \cora{}; and \enquote{Diabetes Mellitus, Experimental},
\enquote{Diabetes Mellitus Type 1}, and \enquote{Diabetes Mellitus Type 2} for
\pubmed{}). Edges between papers are weighted by the cosine similarity between
their abstracts seen as bag of words. I consider only the largest connected
component of each graph, and for \pubmed{}, I further restrict it to its
3-core\footnote{The main 3-core is the largest set of nodes that have at least
3 neighbors in that same set.}. Then, I built 10 nearest neighbors graphs from
4500 images of digit chosen at random in the USPS dataset and from 4500 news
article of Reuters Corpus Volume 1 chosen at random among those having a single
category among the 6 most frequent ones, following the experimental setting
described in~\autocite{WTA13} (which also describes how to ascribe weight to
the edges based on a Gaussian kernel of the Euclidean distance between
instances). Finally, in \imdb{}~\autocite{ImdbData07}, nodes are movies
released in the United States between 1996 and 2001, they are linked if they
share a production company, and the weight of an edge is the number of such
shared production companies. Movies belong to two classes, depending of whether
they grossed more than 2 million dollars on their opening
weekend\footnote{\url{http://netkit-srl.sourceforge.net/data.html}}.  The size
of those 6 weighted graphs is displayed in~\autoref{tab:datasets}.

\begin{table}[hb]
  \centering
  \caption{Datasets statistics}
  \label{tab:datasets}
  \begin{tabular}{lrr}
    \toprule
    Dataset     & $|V|$     & $|E|$      \\
    \midrule
    \imdb{}     & \np{1126} & \np{20282} \\
    \citeseer{} & \np{2110} & \np{3668}  \\
    \cora{}     & \np{2484} & \np{5068}  \\
    \pubmed{}   & \np{4201} & \np{21042} \\
    \usps{}     & \np{4500} & \np{33121} \\
    \rcv{}     & \np{4500} & \np{32715} \\
    \bottomrule
  \end{tabular}
\end{table}

Later I would like to add another dataset based on airport traffic. Basically,
from the list of all flights which departed from or arrived at a US airport in
2016, I would build a graph where nodes are airports, connected by an edge
weighted by the number of flights between those two airports. Then I would
divide airports between regional and worldwide ones (based on the data), giving
a binary label the to the nodes. My assumption is that most regional airports
are connected to others regional ones (the same being true for international
ones), meaning there should be a small number of $\Phi$ edges.

\section*{Setup}
\label{sec:Protocol}

I performed the following experiment. For each dataset, I chose 5 sizes of
training set: $2.5\%$, $5\%$, $10\%$, $20\%$ and $40\%$. For each size of
training set, I also chose 5 perturbation levels $p_0$: $0\%$ (that is, no
perturbation at all), $2.5\%$, $5\%$, $10\%$ and $20\%$. In the results section,
the yellow line correspond to the number of $\Phi$ edges (without taking weights
into account) and as expected, it increases with $p_0$.

When we say that the perturbation level is $p_0$ (for instance $p_0=5\%)$, it
means that the sign of each node in the training set is flipped with probability
$p_0$. Although I didn't compute the number of such flipped signs, it is a
random variable following a binomial distribution of parameters (size of the
training set, $p_0$) and is therefore concentrated around its mean.
\iffalse
the actual probability $p_i$ of each node $i$ to have its sign flipped depends of
its degree $d_i$ in the following way. Denoting $\bar{d}$ the average degree and
$\Delta$ the maximum degree, we first map linearly degrees in the interval $[0,
\bar{d}]$ to probability in $[0, p_0]$ and degrees in the interval $[\bar{d},
\Delta]$ to probability in $[p_0, 2p_0]$. Then we rescale the resulting
probabilities so that their average is $p_0$.
\fi

Once the training set has been fixed and the labelling perturbed, we repeat the
following procedure 12 times:\footnote{Unfortunately, this does let us know
which part of the variance is due to the choice of the training set and which
part is due to the perturbation (but based on the $0\%$ results, I would say the
training set has only a small influence on the overall variance).}

We draw 15 random spanning trees\footnote{I also plotted the results for a
minimum spanning tree (MST) but on all the graphs there is a unique one, hence
we cannot aggregate results over multiple MST.} and for each of these trees, the training
set is presented with 13 different orders during the online phase, which
introduce a variance in the guilt coefficient computed for nodes of the training
set\footnote{Because of the way the code is organized, it's easier to do the
same for \shazoo{}, although in that case, it has naturally no effect on the
results.}. We predict the remaining nodes in a batch fashion. We use majority
vote on each node to aggregate these $15\times 13 = 195$ predictions into a
single one. Finally we evaluate the performance by the number of times the
predicted label of a node differs from the real label, divided by the size of
the testing set (more shortly, the mistake rate).

The two competitors are label propagation\footnote{\lprop{} implementation
is very fast, as it only requires a constant number of multiplications between
the sparse weighted adjacency matrix of the graph and the binary vector of
labels.} and \wta{}~\autocite{WTA13} running on the same trees.

\section*{Results}
\label{sec:Results}

One immediate comment on the results showed in the following pages is that the
number of mistakes is not monotonically increasing with the perturbation level.
First, this does not happen when I compute the mistakes with respect to the
perturbed labelling (i.e. the labelling actually showed to the learner) and
second, I suspect this is an artefact due to the number of repetitions ($12$) not
being large enough. Indeed, if we take for instance the case of the \citeseer{}
dataset with a $2.5\%$ training set and perturbation going from $2.5\%$ to $5\%$,
and repeat the same experience 90 times, we get the following results:
\begin{center}
  \begin{tabular}{lcc}
    \toprule
    Perturbation & $2.5\%$ & $5\%$ \\
    \midrule
    15*\shazoo{}+RTS & $13.73 \pm 2.22$\% & $14.09 \pm 2.42$\% \\
    15*\rta{}+RTS & $13.56 \pm 1.91$\% & $13.82 \pm 2.25$\% \\
    \bottomrule
  \end{tabular}
\end{center}
which shows that the difference in number of mistakes between the two levels of
perturbation is well within one standard deviation, making comparison between few
measurements misleading.
\medskip

\iffalse
Another disturbing fact is that as the training size increases, the performance
of both \shazoo{} and \rta{} does not to seem to improve (and even decreases
sometimes, for instance on \cora{} between $20\%$ and $40\%$ at the $0\%$
perturbation level).  I have no explanation for that so far, although I'm sure
(by looking at the execution time) that both algorithm indeed used a larger
training set during the experiments.
\fi

As for the performance themselves, there are a few general trends. \rta{} and
\shazoo{} are close to each other and there is no clear situation where one is
better than the other. Furthermore, \lprop{} is better, although depending of
the dataset, \rta{} and \shazoo{} can get more competitive. Overall, \wta{} is
clearly the worst.  Here are some percentile of mistake rates, over the all
$12\times 5\times 5\times 6=1800$ point computed during the
experiments\footnote{12 repetitions, 5 training size, 5 perturbation level, 6
datasets.}:

\begin{center}
\begin{tabular}{lccccc|c}
  \toprule
  & $10\%$       & $25\%$       & median       & $75\%$       & $90\%$       & mean         \\
  \midrule
  \lprop{}    & \np{3.04}\% & \np{10.41}\% & \np{13.08}\% & \np{18.38}\% & \np{23.74}\% & \np{14.08}\% \\
  \rta{}      & \np{3.95}\% & \np{10.68}\% & \np{13.55}\% & \np{19.67}\% & \np{25.94}\% & \np{15.01}\% \\
  \shazoo{}   & \np{4.07}\% & \np{10.74}\% & \np{13.72}\% & \np{19.92}\% & \np{26.32}\% & \np{15.19}\% \\
  \wta{}      & \np{4.80}\% & \np{11.42}\% & \np{14.72}\% & \np{22.04}\% & \np{28.15}\% & \np{16.42}\% \\
  \bottomrule
\end{tabular}
\end{center}

For instance, in half of the experiments, \rta{} made less than \np{13.55}\%
mistakes. Even though the average performance of \rta{} and \shazoo{} is close,
we can reject the hypothesis of equal means with a $p$-value of \np{5.856e-8}.
The distributions of the mistake rate over the full 1800 experiments are showed
in \autoref{fig:mistake}.

\begin{figure}[htpb]
  \centering
  \includegraphics[width=1.0\linewidth]{res_shazoo_all.pdf}
  \caption{Distribution of mistake rate. \label{fig:mistake}}
\end{figure}


After the results showed in the next 6 pages,  I also performed smaller scale
experiments to study the variance of the results with respect to number of trees
and number of training set presentation orders. For that, I draw a single
training set of size 10\%, and perturb it once with $p_0=10\%$. Then I draw
\np{29} random spanning trees and, for each tree, I present the training size in
\np{117} different order. I now have $29\times 117$ predictions for each node in
the testing set.

Then let me fix an odd number $k$ of trees and an odd number $\ell$ of
presentation orders. I choose 20 times $k$ trees (among the 29), and for each
such set of trees, I choose 30 times $\ell$ training order (among the 117
possible). For each of these 600 choices, I compute the mistake rate of the
majority vote of their $k\times \ell$ combined predictions. For a given pair
$(k, \ell)$, I then average mistakes rate over the 600 trials. Finally, I do
that for $k \in \{1,5,9,13,17,21,27\} \times \ell \in \{13,  17,  21,  25,  29,
33,  37,  43,  47,  51,  55,  59, 63,  67,  71,  75,  79,  83,  87,  93,  97,
101, 105, 109, 113, 117\}$.

As showed in Figures~\ref{fig:imdb_var}--\ref{fig:rcv1_var}, adding more trees
markedly decreases the mistakes rate and the variance. On the other hand, the
number of training set presentations seems to have little to no effect (except
maybe in \citeseer{}---see \autoref{fig:citeseer_var}).

\iffalse
\newpage
\newgeometry{hmargin=0cm,top=1.2cm,bottom=1cm}
\pagestyle{empty}
\begin{figure*}[!p]
	\centering
	\begin{subfigure}[b]{.945\textwidth}
		\centering
		\includegraphics[height=.30\textheight]{sz_treevar_imdb.pdf}
		\caption{\imdb{}, 10\% training size and 10\% perturbation}
	\end{subfigure}
	\begin{subfigure}[b]{.945\textwidth}
		\centering
		\includegraphics[height=.30\textheight]{sz_treevar_citeseer.pdf}
		\caption{\citeseer{}, 10\% training size and 10\% perturbation}
	\end{subfigure}
	\begin{subfigure}[b]{.945\textwidth}
		\centering
		\includegraphics[height=.30\textheight]{sz_treevar_cora.pdf}
		\caption{\cora{}, 10\% training size and 10\% perturbation}
	\end{subfigure}
  \caption{Mistake rate (and its standard deviation) as the number of trees
  drawn on the graph increases. \label{fig:tree_variance}}
\end{figure*}
\clearpage
\restoregeometry

\newpage
\newgeometry{hmargin=0cm,top=1.2cm,bottom=1cm}
\pagestyle{empty}
\begin{figure*}[!p]
	\centering
	\begin{subfigure}[b]{.945\textwidth}
		\centering
		\includegraphics[height=.30\textheight]{sz_batchvar_imdb.pdf}
		\caption{\imdb{}, 10\% training size and 10\% perturbation}
	\end{subfigure}
	\begin{subfigure}[b]{.945\textwidth}
		\centering
		\includegraphics[height=.30\textheight]{sz_batchvar_citeseer.pdf}
		\caption{\citeseer{}, 10\% training size and 10\% perturbation}
	\end{subfigure}
	\begin{subfigure}[b]{.945\textwidth}
		\centering
		\includegraphics[height=.30\textheight]{sz_batchvar_cora.pdf}
		\caption{\cora{}, 10\% training size and 10\% perturbation}
	\end{subfigure}
	\caption{Mistake rate (and its standard deviation) as the number of training
    set presentations increases.\label{fig:batch_variance}}
\end{figure*}
\clearpage
\restoregeometry
\fi

\newpage

\pagestyle{empty}
\includepdf[pages=-]{res_shazoo_imdb.pdf}
\includepdf[pages=-]{res_shazoo_citeseer.pdf}
\includepdf[pages=-]{res_shazoo_cora.pdf}
\includepdf[pages=-]{res_shazoo_pubmed.pdf}
\includepdf[pages=-]{res_shazoo_usps.pdf}
\includepdf[pages=-]{res_shazoo_rcv1.pdf}


\newpage
\pagestyle{empty}
\newgeometry{hmargin=2.0cm,top=0.6cm,bottom=0.6cm}
\begin{figure}[!t]
  \centering
  \includegraphics[height=.46\textheight]{sz_allvar_imdb.pdf}
  \caption{Variance in \imdb{} at 10\% training size with 10\% perturbation level.\label{fig:imdb_var}}
\end{figure}
\begin{figure}[!b]
  \centering
  \includegraphics[height=.46\textheight]{sz_allvar_citeseer.pdf}
  \caption{Variance in \citeseer{} at 10\% training size with 10\% perturbation level.\label{fig:citeseer_var}}
\end{figure}

\begin{figure}[!t]
  \centering
  \includegraphics[height=.46\textheight]{sz_allvar_cora.pdf}
  \caption{Variance in \cora{} at 10\% training size with 10\% perturbation level.\label{fig:cora_var}}
\end{figure}
\begin{figure}[!b]
  \centering
  \includegraphics[height=.46\textheight]{sz_allvar_pubmed.pdf}
  \caption{Variance in \pubmed{} at 10\% training size with 10\% perturbation level.\label{fig:pubmed_var}}
\end{figure}

\begin{figure}[!t]
  \centering
  \includegraphics[height=.46\textheight]{sz_allvar_usps.pdf}
  \caption{Variance in \usps{} at 10\% training size with 10\% perturbation level.\label{fig:usps_var}}
\end{figure}
\begin{figure}[!b]
  \centering
  \includegraphics[height=.46\textheight]{sz_allvar_rcv1.pdf}
  \caption{Variance in \rcv1{} at 10\% training size with 10\% perturbation level.\label{fig:rcv1_var}}
\end{figure}
\clearpage
\restoregeometry

\begingroup
\setstretch{0.9}
\setlength\bibitemsep{2pt}
\printbibliography
\endgroup
\end{document}
