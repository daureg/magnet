\section{Problem definition}

We observe users and their similarities along different known directions. We
assume that these links arise from users' hidden profile and our goal is to
recover these profiles to answer queries like: are users $i$ and $j$ similar
along direction $x$? To better model real situations, we introduce a further
bias: along all (\comment{relevant?}) directions, there are only a small number
of users clusters.

Formally, there are $n$ users $u_1, \ldots, u_n \in \mathbb{R}^d$ (\comment{and
$||u_i|| = \;1 \forall i$ ?}). Directions are also unit vector of
$\mathbb{R}^d$. Thus, $u_i\cdot x$ literally measures the alignment of $u_i$
with the direction $x$. For instance, $x$ may be the features of an item and
$u_i\cdot x$ a rating. The similarity between $u_i$ and $u_j$ along $x$ is
$sim(u_i, u_j) = \left( |u_i\cdot x - u_j\cdot x| < t \right) \in \{0, 1\}$,
where $t \in \mathbb{R}^+$ is a threshold.

As input, we are given a set of positive examples $\{(u_{i_k}, u_{j_k},
x_k)\}_{k=1\ldots D}$ that we represent as a multigraph: nodes are users, edges
are labelled by the direction along which they represent similarity.


\section{Settings}

In the following we consider batch setting, with a fully observed graph as test
set. Yet for the sake of scalability, online setting may be more appropriate:
either we see the whole graph but edge labels are given one by one or, more
challenging, both edges and their labels are revealed at each time step.

\section{Applications, use cases}

As defined, the problem is rather general. Here we provide two concrete
examples of application. The first is recommender systems. Say we observe a
ratings matrix, where rows are users and columns items. Items are characterised
by a known vector of $d$ features. Users react to these features according to
their hidden profile. We build the train set from the observed entries of the
rating matrix and later fill missing entries by answering similarity queries
(although if ratings from both users are missing, then we can only say whether
they are equal or not, but not their value)

The second is social networks. Here users have also features (some of them are
known, like demographics) and they like contents. \comment(and then what?)

\section{Related work}

\begin{itemize}
	\item bandits \autocites{ClusterBandit14}{Li2015}
	\item Online matrix completion \autocite{OnlineMatrix15}
	\item label propagation \autocite{LabelPropa03}
	\item multilabel classification \autocite{Madjarov2012}
	\item Recovering $d$ different metric spaces whose intersection form the
		observed social multiplex in almost linear time with bounded distortion
		\autocite{Abraham2012a}. Assume there are $K$ social categories
		modelled by Euclidean spaces $\mathcal{D}_i$. Users in there are near
		uniformly distributed and categories have small local correlation: \[
			\forall i\neq i', \; \forall r, r' = O(polylog(n)),\, \forall u,
			u',\, |\mathcal{B}_i(u, r) \cap \mathcal{B}_{i'}(u', r')| \leq O(\log n) \]
		These spaces give rise to small world networks $\mathcal{G}_i$ with
		edge probability $\propto \mathcal{D}_i(u, v)^{-d}$ and we observe the
		real network $\mathcal{G} =\bigcup_i \mathcal{G}_i$. From
		$\mathcal{G}$, the proposed algorithm recovers in $n \mathrm{polylog} n$ time
		a bounded approximation $\mathcal{D}'_i$ of all $\mathcal{D}_i$ \[
			\sigma \mathcal{D}_i(u, v) \leq \mathcal{D}'_i(u, v) \leq \delta
			\mathcal{D}_i(u, v) + \Delta \]
	\item online similarity/dissimilarity prediction on graph \autocite{Gentile2013}
	\item coclustering \autocite{Dhillon2001} (except there's no notion of
		user/item features there)
\end{itemize}

\section{Approaches}

\subsection{Baselines}
A natural way to recover the $u$ vectors is to solve the following
optimization problem:
\begin{equation}
	\min \sum_{k=1}^D \left( (u_{i_k} - u_{j_k} ) \cdot x_k \right )^2
	\label{eq:optim}
\end{equation}
which can also be expressed as a quadratic program, denoting by $\tilde{u}$ the
concatenation of all users vectors:
\begin{equation*}
	\tilde{u}^T \mathbf{Q} \tilde{u} + c^T \tilde{u}
\end{equation*}
A caveat of this method is lack of scalability. Furthermore, we need to impose
constraints to avoid the trivial solution $\tilde{u} = 0$. If we constrain $u$
to be unit norm, the problem is not convex. An alternative is to fix some
values: either we already know some user profiles or some dimension.

Another immediate idea is to solve $d$ separate problems, one across each
dimension, and somehow combine the results. Yet this is also not very
satisfactory, as it looses all cross dimension information.

\subsection{Approximation}

To simplify the problem, we could transform it from vector to scalar by using
random projection. Indeed, let $a \in \{-1,1\}^d$ be a vector drawn uniformly
at random. Then $\mathbf{E}_{a \in \{-1,1\}^d}\left[(u\cdot a)\times (x \cdot
a)\right] = u\cdot x$ and we get a weighted version of \eqref{eq:optim}:
\begin{equation*}
	\min \sum_{k=1}^D (x_k\cdot a)^2 \left( (u_{i_k} - u_{j_k} ) \cdot a \right )^2
	\label{eq:random-optim}
\end{equation*}

Then we would like to take advantage of the graph structure as well (which is
implicit in \eqref{eq:optim}, as each edge contributes one term to the sum).
For instance by sparsifying the graph with relevant trees. Yet one should keep
in mind that the input is a multigraph, whose edges labels are vectors and not
mere weights. One idea could be to cluster the set of all directions of the
training set using $k$-means and work on these $k$ induced graphs.

\section{First practical steps}

\comment{This part is quite messy…}

\subsection{datasets}

\subsubsection{real}

MovieLens 1M
\href{http://files.grouplens.org/datasets/movielens/ml-1m-README.txt}%
{http://files.grouplens.org/datasets/movielens/ml-1m-README.txt} contains
demographics information about 6040 users, namely age bracket, job (converted
to median income), gender and ZIP code (converted to population density),
whereas hetrec2011-movielens-2k
\href{http://ir.ii.uam.es/hetrec2011/datasets/movielens/readme.txt}%
{http://ir.ii.uam.es/hetrec2011/datasets/movielens/readme.txt} contains the
following information on a subset of movies: release year, nationality
(converted to 0: US,UK,CA,AUS; 1: EU; 2: others), number of critics ratings on
rottentomatoes\footnote{\href{http://www.rottentomatoes.com/}{rottentomatoes.com}},
average critics ratings, number of audience ratings, average ratings. In
addition, for some of the pair (user, movie), we have a integer rating between
1 and 5.

To match the dimension between users and movies features, I later ditched
demographics attributes and instead fit to each user ratings a linear
model\footnote{\href{http://scikit-learn.org/stable/modules/generated/sklearn.linear_model.Ridge.html}%
{\texttt{sklearn.linear\_model.Ridge}}}, which add a dimension to store the
intercept parameter.

R6A - Yahoo! Front Page Today Module User Click Log Dataset
\href{http://webscope.sandbox.yahoo.com/catalog.php?datatype=r}%
{\url{http://webscope.sandbox.yahoo.com/catalog.php?datatype=r}}.
While it was originally very rich, we only observe the result of a heavy pre
processing described in \autocite{YahooNews09}. Users and articles are
described by 6 features and there is also binary feedback: article cliked or
not.

Unfortunately, being real data, they don't fit well our assumptions.
Specifically, computing dot products between all possible user/items pairs,
there is no threshold that discriminate between similar ratings or not. Thus
we turn to synthetic data.

\subsubsection{synthetic}

We generate $n$ users and $m$ items as uniform random unit vectors. Then we
discretize all $u_i\cdot x_j$ to get ratings between 1 and 5 and finally we
set a proportion $t$ of ratings to 0 (unobserved). \comment{This doesn't take
the clustering assumption into account}.

We then sample 50\% of the non zero entries. Items hit by this process are in
the train set while the others will be used for test. Users hit will later be
divided into train and test set, while the others are excluded (since we
didn't observe anything about them, we can't learn anything about them either).
Then we infer user profiles from available data (\comment{this time I use a
noisy version of the original $u$ vectors, which is admittedly an easier
setting}).

Now we can build the input multigraph, where there is an edge between two users
along an item if they gave it the same rating. As there are many possible such
edges and we initially want to iterate fast, I decided to discard some of them.
Basically I consider only edges between users that are close, as measured by
the distance between their common ratings vector. \comment{A more principled
way would be to consider all edges and discard those whose dot product
difference is above a given threshold}.

Finally we solve \eqref{eq:optim} and get the results shown in
\autoref{fig:syn_accuracy}.

\begin{figure}[hb]
	\centering
	\includegraphics[width=1.0\linewidth]{syn_accuracy.png}
	\caption{.\label{fig:syn_accuracy}}
\end{figure}
