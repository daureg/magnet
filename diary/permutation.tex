\documentclass[a4paper,final,notitlepage,11pt,svgnames]{scrartcl}
\usepackage{ifxetex}
\ifxetex
\usepackage{polyglossia}
\setmainlanguage{english}
\usepackage{xltxtra}
\setmainfont{Linux Libertine O}
\fontspec[Numbers={OldStyle}]{Linux Libertine O}
\else
\usepackage[english]{babel}
\usepackage[utf8]{inputenc}
\fi

% \usepackage{aaltologo}
% \renewcommand{\thesubsection}{\alph{subsection}}
\usepackage{algorithm}
\usepackage[noend]{algpseudocode}
\newcommand*\Let[2]{\State #1 $\gets$ #2}
\def\algorithmautorefname{Algorithm}
\usepackage{amsmath}
% \usepackage{amsfonts}
\usepackage{amssymb}
\usepackage[autolanguage]{numprint}
% \usepackage{balance}
\usepackage{tabularx}
\usepackage{tabulary}
\usepackage{multirow}
\usepackage{booktabs}
% \usepackage{caption}
% \usepackage{chngpage}
% \usepackage{colortbl}
\usepackage{graphicx} 
\usepackage{microtype}
% \usepackage{natbib}
% \usepackage{pgfplots}
% \usepackage{pgfplotstable}
\usepackage[strict]{csquotes}
% \usepackage{caption}
% \usepackage{subcaption}
\usepackage{xcolor}
\usepackage{varioref}

% \DeclareMathOperator*{\argmax}{argmax}
% \newcommand{\argmax}[1]{\arg\underset{#1}{\max}\;}
% \newcommand{\equiv}{ \Leftrightarrow}
% \newcommand\numberthis{\addtocounter{equation}{1}\tag{\theequation}}
% \newcommand{\vnorm}[1]{\left|\left|#1\right|\right|}
% \renewcommand{\thesubsection}{\alph{subsection}}

% \pgfplotsset{height=8cm,compat=newest}
% \pgfplotstableset{
% every head row/.style={before row=\toprule,after row=\midrule},
% every last row/.style={after row=\bottomrule},
% every even row/.style={before row={\rowcolor{white}}},
% }
% \tikzexternalize% activate externalization!
% \tikzset{cost/.style={inner sep=0.5mm,align=center},
% traceback/.style={->,>=stealth,Blue}, }
% \usepgfplotslibrary{external}
% \usetikzlibrary{calc,arrows}

\usepackage[citestyle=authoryear-comp,bibstyle=ieee,isbn=false,maxcitenames=1,mincitenames=1,sorting=nyt,backend=biber,defernumbers=true]{biblatex}

\AtEveryBibitem{
   \clearfield{arxivId}
   % \clearfield{booktitle}
   \clearfield{doi}
   \clearfield{eprint}
   \clearfield{eventdate}
   \clearfield{isbn}
   \clearfield{issn}
   % \clearfield{journaltitle}
   \clearfield{month}
   % \clearfield{number}
   % \clearfield{pages}
   \clearfield{series}
   \clearfield{url}
   \clearfield{urldate}
   \clearfield{venue}
   % \clearfield{volume}
   \clearlist{location} % alias to field 'address'
   \clearlist{publisher}
   \clearname{editor}
}
\addbibresource{../../../biblio/magnet.bib}
% From https://www.writelatex.com/630785fxzchp
% \usepackage[hmargin=2cm,vmargin=2.5cm]{geometry}
\usepackage{fancyhdr}
\newcommand{\userName}{Géraud Le Falher}
\newcommand{\institution}{Inria Lille}
% \pagestyle{fancy}
\rhead{\workingDate}
\chead{\textsc{Research Diary}}
\lhead{\textsc{\userName}}
\rfoot{\textsc{\thepage}}
\cfoot{\textit{Last modified: \today}}
\lfoot{\textsc{\institution}}

\usepackage{wrapfig}
\newcommand{\univlogo}{%
  \noindent % University Logo
  \begin{wrapfigure}{r}{0.3\textwidth}
    \vspace{-2\baselineskip}
    \begin{center}
	    \includegraphics[width=4cm]{../../admin/inria_logo.pdf}
    \end{center}
  \end{wrapfigure}
}

\usepackage{hyperref}
\hypersetup{%
    % draft,    % = no hyperlinking at all (useful in b/w printouts)
    colorlinks=true, linktocpage=true, pdfstartpage=3, pdfstartview=FitV,%
    % uncomment the following line if you want to have black links (e.g., for printing)
    %colorlinks=false, linktocpage=false, pdfborder={0 0 0}, pdfstartpage=3, pdfstartview=FitV,%
    breaklinks=true, pdfpagemode=UseNone, pageanchor=true, pdfpagemode=UseOutlines,%
    plainpages=false, bookmarksnumbered, bookmarksopen=true, bookmarksopenlevel=1,%
    hypertexnames=true, pdfhighlight=/O,%nesting=true,%frenchlinks,%
    urlcolor=Chocolate, linkcolor=DodgerBlue, citecolor=LimeGreen, %pagecolor=RoyalBlue,%
    %urlcolor=Black, linkcolor=Black, citecolor=Black, %pagecolor=Black,%
}

\usepackage{pifont}

\newcommand{\inci}{\ensuremath{\mathcal{F}}}

\title{Counting permutations under constraints}
\author{Géraud Le Falher}
\begin{document}
\maketitle

\section*{Setting \& Notations}

Let $n \in \mathbb{N}$ and $\inci{}$ be a \textbf{set}\footnote{thus not
containing any repetition} of \emph{forbidden edges}, where an edge is defined
as an (directed) pair of distinct natural numbers smaller than $n$, e.g.  $(1,
3)$. If a permutation $\sigma \in \mathfrak{S}_n$ of $n$ includes at least one
forbidden edge from \inci{}, we say $\sigma$ is forbidden by \inci{}.  For
instance, assuming $n=8$ and $\inci{} = \{(1,3), (4,6)\}$, $\sigma =
(8 \rightarrow 1 \rightarrow 3 \rightarrow 7 \rightarrow 4 \rightarrow 6
\rightarrow 3 \rightarrow 2)$ is forbidden because it includes $(4,6)$ and $(1, 3)$.

Our goal is to compute the number of permutations not forbidden by \inci{}: \[
f(\inci{}) = n! - \left|\{\sigma \in \mathfrak{S}_n; \sigma \; \text{is forbidden
by}\; \inci{}\}\right| \]

As a special case, let us denote $|S_0| = n!$, $|S_1| = f(\inci{})$ where
$|\inci{}| = n-1$ and more generally $|S_i| = f(\inci{})$ where $|\inci{}| =
i(n-1)$. The question is then how many times can we be told than $n-1$ edges do
not exist before $|S_i| \leq 1$.

\section*{Fun with Combinatorics!}

First, assume $\inci{} = \{(u,v)\}$. Let's build a forbidden permutation.
Because $v$ should always follows $u$, we can pretend they form a single node
and collapse them. Left with $n-1$ nodes, we can freely order them in one of
the $(n-1)!$ possible way and be guaranteed this will include $u \rightarrow
v$.

Now if $\inci{} = \{(u,v), (w,x)\}$, how many forbidden permutations are there?
Well $(n-1)!$ permutations include $(u,v)$, $(n-1)!$ include $(w, x)$ but we
are double counting those with both $(u,v)$ and $(w,x)$. By the same collapsing
argument applied to the two pairs of nodes, there are $(n-2)!$ permutations
include the two edges. Note furthermore that this still holds if the edges are
not node disjoint (e.g. $\inci{} = \{(u,v), (v,w)\}$).

Armed with the fact that there are $(n-i)$ permutations which include $i$
forbidden edges\footnote{when each node can only appear once at the head of an
edge}, we can use the inclusion-exclusion principle to compute $|S_1|$
\begin{align*}
	|S_1| & = n! - \sum_{i=1}^{n-1} (-1)^{i+1} \binom{n-1}{i} (n-i)! \\
          & = \sum_{i=0}^{n-1} (-1)^i \frac{(n-1)!}{i!(n-1-i)!} (n-i)! \\
          & = (n-1)! \sum_{i=0}^{n-1} (-1)^i \frac{(n-i)}{i!} \sim \frac{n!}{e}
\end{align*}

\bigskip

In $S_1$, each node has only one forbidden successor. In $S_2$ it has two.
However, some combinations counted by the binomial coefficients in the formula
cannot appear. For instance, while we are still double counting $[(0,1),
(1,2)]$, it makes not sense to subtract $[(0,1), (0,2)]$ as it is not feasible
in any permutation anyway. Therefore we need to replace the binomial
coefficients $\binom{|\inci{}|}{i}$ by $D_i$, which is the number of sequences
of $i$ edges from \inci{} such that each node appear at most once at the head
of an edge.

For instance, if $n=4$ and $\inci{} = \{(0,1),(0,2),(1,2),(2,3)\}$

\begin{center}
	\begin{tabular}{lll}
		\toprule
		$D_1 = 4$ & $D_2 = 4$                      & $D_3 = 1$             \\
		\midrule
		$[(0,1)]$   & $[(0,1),(1,2)], [(0,1),(2,3)]$ & $[(0,1),(1,2),(2,3)]$ \\
		$[(0,2)]$   & $[(0,2),(1,2)]$                &                       \\
		$[(1,2)]$   & $[(1,2),(2,3)]$                &                       \\
		$[(2,3)]$   &                                &                       \\
		\bottomrule
	\end{tabular}
\end{center}

\dingline{96}
\begin{center}
       % \includegraphics[height=.0666\textheight]{dinol.pdf}\hspace{2em}
       % \includegraphics[height=.0666\textheight]{dinol.pdf}\hspace{2em}
       % \includegraphics[height=.0666\textheight]{dinol.pdf}\hspace{2em}
       % \includegraphics[height=.0666\textheight]{dinor.pdf}\hspace{2em}
       % \includegraphics[height=.0666\textheight]{dinor.pdf}\hspace{2em}
       % \includegraphics[height=.0666\textheight]{dinor.pdf}
\emph{Beginning of the handwavy part ;)}
\end{center}

More generally for $S_2$, denoting $|\inci{}|=2(n-1)=m$, $D_0 = 1$ (by
convention) $D_1 = m$ (as any single edge is admissible), $D_2 =
\frac{(m-1)(m-2)}{2}$, $D_3 = \frac{(m-2)(m-3)(m-4)}{6}$ and so on (although
I'm not sure exactly why). Empirically, $D_i = \binom{m+1-i}{i}$, even though
it gets a bit off as $i$ gets closer to $n-1$. Especially, $D_{n-1} = \lceil
\frac{n}{2} \rceil$. Still
\begin{align*}
	|S_2| & \approx \sum_{i=0}^{n-1} (-1)^i \binom{2n-1-i}{i} (n-i)! \\
       & = \sum_{i=0}^{n-1} \frac{(-1)^i}{i!} \underbrace{\frac{(2n-1-i)!(n-i)!}{(2(n-1)-i)!}}_{a_i} \\
\end{align*}

Removing the $-1$ of $a_i$ for clarity, we see that $a_0 = n!$ and $a_n =
n!$\footnote{even though it doesn't appear in the sum}, while the maximum is
reached at $i=\lceil \frac{2n}{3} \rceil$\footnote{Again no matter how
convinced I am it's true, that's a leap of faith}: let $n=3p$, \[
	a_i = \frac{(2\cdot3p - 2p)!(3p-2p)!}{2(3p-2p)!} =
	4p\cdots\underbrace{3p}_{n} \cdot 1 \cdots1 \cdot p\cdots 1
\]

\end{document}
