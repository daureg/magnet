The idea behind clustering is, given a set of objects, to gather similar ones
into clusters so that all objects belonging to one cluster are similar to each
other while being dissimilar to objects from all the other clusters.

One way to formalize this problem is to consider objects as the nodes of a
graph, whose edges weight encode similarity. Namely, weights sign denote
similarity or not, while their absolute value indicate the strengh of the
relationship. 

The objective can then be formulated in two way, either maximizing agreements
(the number of $+$ edges inside clusters and $-$ edges across clusters) or
minimizing disagreement (the number of $-$ edges within clusters and $+$ edges
between clusters):

\maxa{}
\begin{equation}
		\max_{\mathcal{S}} \sum_{(u,v)}
		c(w_{uv}) \iverson{w_{uv}>0} \iverson{\mathcal{S}(u) = \mathcal{S}(v)} +
		c(w_{uv}) \iverson{w_{uv}<0} \iverson{\mathcal{S}(u) \neq \mathcal{S}(v)}
		\label{eq:maxa}
\end{equation}

\mind{}
\begin{equation}
		\min_{\mathcal{S}} \sum_{(u,v)}
		c(w_{uv}) \iverson{w_{uv}<0} \iverson{\mathcal{S}(u) = \mathcal{S}(v)} +
		c(w_{uv}) \iverson{w_{uv}>0} \iverson{\mathcal{S}(u) \neq \mathcal{S}(v)}
		\label{eq:mind}
\end{equation}

where $\mathcal{S}$ is a clustering and $\mathcal{S}(u)$ is the cluster index
of $u$ and $c$ is a cost function.

\section{A bit of history}
\label{sub:histo}

\Textcite{Ben-Dor99} consider an similar problem in the context of gene
expression.  They assume there exists a perfect clustering, and receive as
input a complete similarity matrix, which is corrupted by measurement error.
They give an $O(n^2(\log n)^c)$ algorithm that recover the planted disjoint
cliques with high probability.
Later, \textcite{Shamir02} study the closely related \textsc{Cluster editing}
problem: how few edges to add and remove in the input graph to turn it into a
vertex-disjoint union of cliques?\footnote{A recent reference on exact
parametrized decision version of this problem is \autocite{Fomin2014}.} They
showed the problem is \npc{}, even if
the number of clusters $p\geq2$\footnote{The reduction is from 3-exact 3 cover,
see \autocite[Theorems 1, 2 and Corollary 1]{Shamir02}.} is set beforehand
and provide a $0.878$ in weighted $p=2$ case using a standard SDP relaxation.
Independently, \textcite{Bansal2002} introduce problems \eqref{eq:maxa} and
\eqref{eq:mind} over complete graph under the name \pcc{}. They give a (large)
constant approximation of \mind{} and a PTAS for \maxa{}.

Two surveys on the topic: \autocites{Becker05}{bonchi2014correlation}
% The number of triangle with a single $-$ edge is a lower bound of the number of disagreements
% \footnote{namely, they defined a distance between two graphs as the symmetric
% difference of their edge sets.  With high probability, the clustering
% returned by their solution but then why not returned the perturbed graph?}

\section{Hardness and approximation}
\label{sec:approx}

\Textcite{Charikar2003} show a 4-approximation on \mind{} for complete graph
using a LP relaxation. On general graph, \mind{} also admits a LP relaxation
in connection with the multicut problem with a $O(\log n)$ approximation
rounding a SDP.

In addition to independently giving the same approximations as
\autocite{Charikar2003}, \textcite{Demaine2006} show the equivalence
between \pcc{} and the weighted multicut problem, which ask for the minimum
weight set of edges whose removal in $G$ disconnect the $k$ pairs $(s_i, t_i)$

\Textcite[Section 4]{Charikar2004} give a $\Omega(\frac{1}{\log n})$
approximation of the \textsc{MaxCorr} problem, which is maximizing
\eqref{eq:maxa} - \eqref{eq:mind} and can be formulated as a quadratic
programming problem solved in polynomial time.

\Textcite{Giotis2006} give an more simple reduction from Graph Min Bisection
When $k$, the number of clusters is fixed, they provide PTAS on complete graph
for \maxa{} in $nk^{O(\epsilon^{-3}\log(\frac{k}{\epsilon}))}$ and \mind{} in
$n^{O(\frac{9^k}{\epsilon^2})}\log(n)$.
They also provide the following summary of results on general graph for fixed
$k$:
\begin{center}
	\begin{tabularx}{\linewidth}{lXX}
		\toprule
		$k$	 & 2 & $\geq 3$ \\
		\midrule
		\maxa{} & 0.878 (improved to 0.884 by \autocite{Mitra2009}) & 0.7666
		\autocite{Swamy2004} \\
		\mind{} & $O(\sqrt{\log n})$ as it reduces to Min 2CNF Deletion for
		which \textcite{min2CNF05} give such an approximation &
		this can be reduced from $k$-coloring, which for any $\epsilon > 0$ is
		\npc{} to approximate within $n^{1-\epsilon}$
		\autocite{InnaproxChroma07} \\
		\bottomrule
	\end{tabularx}
\end{center}

\begin{table}
\begin{tabularx}{\linewidth}{llcXX}
\toprule
							 &                             & $k$   & \mind{}                                                                                   & \maxa{} \\ \midrule
   \multirow{2}{*}{Complete} & unweighted                  &       & 2.06 \autocite{Chawla2014}                                                                & PTAS in $n2^{poly(\frac{1}{\epsilon})}$ \autocite{Giotis2006} \\
   \cmidrule(r){4-5}
							 & weighted                    &       & 1.5 (triangular inequality) \autocite{Chawla2014}                                         & ? \\
   \midrule
   \multirow{3}{*}{General}  & \multirow{2}{*}{unweighted} &       & $O(\log n)$ \autocite{Bansal2002}, optimal under UCG conjecture \autocite{Makarychev2014} & $0.7666$ \autocite{Swamy2004} \\
   \cmidrule(r){4-5}
							 & unweighted                  & $k=2$ & $O(\sqrt{\log n})$ \autocite{min2CNF05}                                                   & $0.884$ \autocite{Mitra2009} \\
   \cmidrule(r){4-5}
							 & weighted                    &       & $O(\log n)$ \textcite{Demaine2006}                                                        & ? \\
\bottomrule
\end{tabularx}
\caption{Best results on various problem.\label{tab:soa}}
\end{table}

Among related problems, \textcite{Ailon2008} also consider \pcc{}.
On complete unweighted graph, they got a 3 approximation using
\textsc{CC-Pivot}. On weighted complete graph, they give two versions, one
that is greedy and a more sophisticated one that involved LP but perform
better. 2.5 approximation with probability constraints ($w_{uv}^+ + w_{uv}^- =
1$) and 2 approximation if we add triangular inequality ($\forall k\;
w_{ij}^- \leq w_{ik}^- + w_{kj}^-$). Using more sophisticated rounding of the
LP, \textcite{Chawla2014} improve these approximation to 2.06 and 1.5
respectively.

Let say weights are bounded and assume w.l.o.g.\@ they lie in $[-1, +1]$.
\Textcite[Section 7]{Bansal2002} describe a linear cost function where an edge
of weight $x$ incurs a cost $\frac{1-x}{2}$ when it's inside a cluster and
$\frac{1+x}{2}$ when it's between two clusters. It is consistent with edges
of integer weight $+1$, $-1$ and $0$, the last ones incurring the same cost
however we put them.

Note that given a graph with weights bounded in absolute value by $M$, we can
always transform them so to respect the probability constraints.

Weak social balance ensure that by forbidding triangles with exactly one
negative edge, complete signed graph are perfectly
clusterable\autocite{davis1967clustering}. Thus triangular inequality can be
seen as a weighted relaxation of it, which explain why it leads to better
approximation.

\section{Applications}
\label{sec:appli}
As stated in \autocite[Section 5]{Demaine2006}:
\begin{itemize}
	\item finding optimal $k$
	\item adding constraints to existing problem
	\item clustering without distance (or with several conflicting
		distance), which can also be also be linked to metric learning.
	\item Visualization of signed social graph \autocite{Luca10}
	\item image segmentation \autocites{Bagon2011}{Kim2011}
	\item duplicate detection, also called entity resolution \autocite{DeDup09}
	\item coreference resolution is solved by \textcite[Section
		2.3]{graphicalCoreference04} using a undirected graphical model on
		which performing inference is equivalent to \pcc{}.
		\Textcite{Elsner2009} compared various heuristics with a bound of the
		optimal solution obtained through SDP relaxation and show that best
		performing ones are within few percents of it, provided they are
		followed by a local search step. The same problem is tackeld
		by \textcite{Chatel14}, albeit with a different approach.
	\item blockmodel membership. For instance, in biology,
		\textcite{Mason2009} analyze a signed co-expression networks of genes
		involved in embryonic stem cells to find which genes are related to
		pluripotency or self-renewal.
\end{itemize}
% section2, Gionis2007
% \autocite{Bonchi2013}
% Thanks to this generality, the technique is applicable to a multitude of
% problems in different domains, including duplicate detection and similarity
% joins [17, 27], biology [11], image segmentation [30] and social networks [12].


\section{Variations and related problems}
\label{sec:related}

\paragraph{recovery under noise}
Already considered in \autocite[Section 6]{Bansal2002}
% summarize that part
\enquote{classic minimum Multicut problem, for which the current
	state-of-the-art algorithm gives a $\Theta(\log n)$ factor approximation
	[GVY93]. This also implies that assuming the Unique Games Conjecture, we
	cannot obtain a constant factor approximation in the worst-case [KV05].}\autocite{Makarychev2014}

But maybe we can do better on average case, which motivate the study of semi
random model, where real graphs are seen as being obtained from the controlled
perturbation of a perfectly clusterable graph. 

\autocite{Joachims2005} \enquote{Our analysis makes three contribution. First, we
	define a model in which we derive finite-sample error bounds for
	correlation clustering. Second, we study the asymptotic behavior of
	correlation clustering with respect to the density of the graph and the
	scaling of cluster sizes. And finally, we propose a statistical test for
	evaluating the significance of a clustering.}

\autocite{Mathieu2010}
% recover clusters given complete similar/dissimilar information which have been
% perturbed by relaxing SDP and clever rounding

\enquote{They consider a generalization of this model where there is an adversary. In
their model, for each edge, with probability $(\frac{1}{2} + \epsilon)$ we do
not flip the initial edge label, and with probability $(\frac{1}{2} -
\epsilon)$ the adversary decides whether to flip the edge label or not. They
give an algorithm that finds a clustering of cost at most $1 + O(n -
\frac{1}{6})$ times the cost of the optimal clustering, as long as $\epsilon
\geq n - \frac{1}{3}$ .  However, these average-case models and algorithms deal
only with the special case of complete graphs, with all edges having unit
cost.}\autocite{Makarychev2014}.


\paragraph{Edge sign prediction}
% better summary
\autocite[Section 6]{Kunegis2009} using global matrix kernel.
While the works mentioned above are combinatorial algorithms over arbitrary,
undirected graph, we can also give additional semantic to edges in directed
graph, such as friendship/foeship, trust/distrust, activator/inhibitor,
secure/malicious interactions or high/low status. Assuming these graphs
display some regularity like social balance, this can be used as bias for
learning algorithms. Some heuristics are local. For instance,
\textcite{Leskovec2010} build a logistic regression model based on degree
feature and frequency of triangle in which each edge is involved.
\Textcite{LowRankCompletion14} consider longer cycle but note that global
approach are more efficient. In particular, inspired by the fact the complete,
balanced adjacency matrix has $k$ disjoint block, they give condition under
which it can be recovered from the observed matrix through low rank matrix
factorization.

\paragraph{Communities detection}
In the context of signed social network, clustering can also be seen as
affecting each to one or more communities
\autocite{Yang2007} based on random walks (that are likely to stay within one
community)
\autocites{Yang2007}{Traag2009}{Doreian2009}{Anchuri2012}{Amelio2013}{Li2013}{Chen14}{Jiang2015}


\paragraph{Online \& active setting}
% better summary
% \emph{Still have to write this part. Is there a link between mistake bound
% and approximation ratio?}
Beside batch setting, one can also consider active
\autocites{Cesa-Bianchi2012b}{Cesa-Bianchi2012a} and online
\autocite{Gentile2013} framework.

\Textcite{mathieu:inria-00455771} give a greedy algorithm that upon vertex
arrival creates a singleton cluster and then merge all pairs of clusters for
which it increases the total number of agreements. For \mind{}, this is
$O(n)$-competitive algorithm and they show such ratio is optimal by exhibiting
an instance\footnote{two positive cliques $A$ and $B$ joined by positive edges
except between $a\in A$ and $\{b_1,\ldots, b_k\}\in B$. Those nodes are given
first and thus form a cluster which yield at least one disagreement every time
one the $n-(k+1)$ remaining vertex is added.} on which any strategy ends up with $n - k$
disagreements whereas optimal cost is $k$. On the \maxa{} side, this greedy
strategy result in a $0.5$-competitive algorithm. If it is randomly mixed with
a \textsc{Dense} variation, it raises up to $0.5+\eta$, still far from the
demonstrated $0.834$ upper bound.

\autocite[Section 5]{Ailon2014} present another active algorithm but it looks
pretty involved so I haven't read it yet: \enquote{algorithms for \mind$[k]$
with sublinear query complexity, but the running time of this solutions is
exponential in $n$}.

\paragraph{Overlapping correlation clustering \autocite{Bonchi2012}} in this setting,
objects are allowed to belong to more than one cluster. Given a complete
weighted graph, output a labelling function $\ell: V \rightarrow 2^{|V|}$ that
minimize: \[C_{occ}(V, \ell) = \sum_{u,v} \left| H(\ell(u), \ell(v)) -
		w_{uv}\right|\] where $H$ is a similarity function between sets of
labels, such as Jaccard similarity or a binary indicator of non empty
intersection. These problems are \nph{} and authors provide approximations
algorithm and map-reduce experimental results.

\paragraph{consensus clustering} 
the goal is to output a clustering which best summarize the given input
clusterings.  It's a restricted case of \pcc{} where the negative weights obey
the triangular inequality.  \Textcite{Gionis2007} give a deterministic 3
approximation and describe a sampling approach suited to large dataset. It was
later improved by \textcite{Bonizzoni2008}, which show that the minimization
version is APX-hard, even with only 3 candidates but give a PTAS with a
$\frac{4}{3}$ approximation for the maximization problem. Experimental
evaluations are conducted by \textcite{Bertolacci07} and \textcite{Filkov08}.

\paragraph{non binary edge labelling}
Here, \enquote{positive} edges are categorical (or colored) and the goal is to
form clusters mostly made up of edges of one color. As a generalisation of
\pcc{}, it's \nph{} and \textcite{Bonchi2012a} give a random approximation
algorithm with a ratio bounded by max degree. They also present a fixed $k$
method and experiment on real world datasets.

% add bipartite (read KDD tuto)
\paragraph{Bipartite \pcc{}}
First 11 approximation \autocite{Amit04}, then 4 approximation using LP
\autocite{Bipartite12} and improved to 3 even for $K$-partite graph \autocite{Chawla2014}

\paragraph{Hypergraphs} \Textcite{Kim2011} show a LP relaxation on hypergraph
and \textcite{Ricatte13} describe a class of hypergraphs that can be reduced
to signed graph.

\paragraph{Other approaches}

\begin{description}
	\item[spectral clustering and embedding] \autocite{SignedEmbedding15}
	\item[exact solution] By casting the problem as a \textsc{MaxSAT}
		instance, one can take advantage of existing solvers to get an exact
		solution on small instances ($n \leq 1000$) \autocite{Berg2015}
\end{description}

\subsection{Scalability}
\paragraph{Distributed computation}
\Textcite{Chierichetti2014} described a distributed approach with the
following performance bound:

\begin{tabular}{lll}
	\toprule
	& \# rounds  & ratio \\
	\midrule
	complete & $O(\log n)$ & $3 + O(\epsilon)$ \\
	general & $O(\log n\log{\Delta^+})$ & $O(\log{n})$ \\
	\bottomrule
\end{tabular}

In the case of unit weight, this can be improved to $O(\log\log n)$ rounds
and $3$ approximation \autocite{Ahn2015}.

% mention unpublished yahoo method
There are other approaches \autocite[Part III]{bonchi2014correlation} one can
look at.

A promising one is to perform Ailon algorithm using multiple cores
\autocite{Pan2014}.

One can also sample the data beforehand, as in \autocite{Bertolacci07}.

\paragraph{Local search}
\Textcite{Bonchi2013} study the problem of finding cluster index consistently
given a single node and making some queries to the adjacency matrix of a
general unweighted graph. They first describe a modification of Ailon
\textsc{QuickCluster} giving a $4\cdot OPT + \epsilon n^2$ approximation in
$O(\frac{1}{\epsilon})$ time after $O(\frac{1}{\epsilon^2})$ time and queries
preprocessing. Then there is an additive $OPT + \epsilon n^2$ additive
approximation in $poly(\frac{1}{\epsilon})$ time after a
$poly(\frac{1}{\epsilon})$ time and $2^{poly(\frac{1}{\epsilon})}$ queries
preprocessing which seems more involved.  Because their approach is localized,
it is trivially parallelized and is independent of the graph size. Therefore
one can obtain a complete clustering in time linear with the number of node to
predict.
\section{Future directions}
Ranging from theoretical to practical ones:

\begin{itemize}
	\item improve bounds, for instance by designing ad-hoc algorithm for
		specific case
	\item improve scalability, for instance using distributed non negative
		matrix factorization, an active area of research
		\autocite{Bhojanapalli2014,Hastie2014,Udell2014}
	\item apply these new methods to real datasets (social network, NLP) and
		interpret domain specific results
\end{itemize}
