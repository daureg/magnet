\documentclass[a4paper,draft,notitlepage,11pt,svgnames]{scrartcl}
\usepackage{ifxetex}
\ifxetex
\usepackage{polyglossia}
\setmainlanguage{english}
\usepackage{xltxtra}
\setmainfont{Linux Libertine O}
\fontspec[Numbers={OldStyle}]{Linux Libertine O}
\else
\usepackage[english]{babel}
\usepackage[utf8]{inputenc}
\fi

% \usepackage{aaltologo}
% \renewcommand{\thesubsection}{\alph{subsection}}
\usepackage{algorithm}
\usepackage[noend]{algpseudocode}
\newcommand*\Let[2]{\State #1 $\gets$ #2}
\def\algorithmautorefname{Algorithm}
\usepackage{amsmath}
% \usepackage{amsfonts}
\usepackage{amssymb}
\usepackage[autolanguage]{numprint}
% \usepackage{balance}
\usepackage{tabularx}
\usepackage{tabulary}
\usepackage{multirow}
\usepackage{booktabs}
% \usepackage{caption}
% \usepackage{chngpage}
% \usepackage{colortbl}
\usepackage{graphicx} 
\usepackage{microtype}
% \usepackage{natbib}
% \usepackage{pgfplots}
% \usepackage{pgfplotstable}
\usepackage[strict]{csquotes}
% \usepackage{caption}
% \usepackage{subcaption}
\usepackage{xcolor}
\usepackage{varioref}

% \DeclareMathOperator*{\argmax}{argmax}
% \newcommand{\argmax}[1]{\arg\underset{#1}{\max}\;}
% \newcommand{\equiv}{ \Leftrightarrow}
% \newcommand\numberthis{\addtocounter{equation}{1}\tag{\theequation}}
% \newcommand{\vnorm}[1]{\left|\left|#1\right|\right|}
% \renewcommand{\thesubsection}{\alph{subsection}}

% \pgfplotsset{height=8cm,compat=newest}
% \pgfplotstableset{
% every head row/.style={before row=\toprule,after row=\midrule},
% every last row/.style={after row=\bottomrule},
% every even row/.style={before row={\rowcolor{white}}},
% }
% \tikzexternalize% activate externalization!
% \tikzset{cost/.style={inner sep=0.5mm,align=center},
% traceback/.style={->,>=stealth,Blue}, }
% \usepgfplotslibrary{external}
% \usetikzlibrary{calc,arrows}

\usepackage[citestyle=authoryear-comp,bibstyle=ieee,isbn=false,maxcitenames=1,mincitenames=1,sorting=nyt,backend=biber,defernumbers=true]{biblatex}

\AtEveryBibitem{
   \clearfield{arxivId}
   % \clearfield{booktitle}
   \clearfield{doi}
   \clearfield{eprint}
   \clearfield{eventdate}
   \clearfield{isbn}
   \clearfield{issn}
   % \clearfield{journaltitle}
   \clearfield{month}
   % \clearfield{number}
   % \clearfield{pages}
   \clearfield{series}
   \clearfield{url}
   \clearfield{urldate}
   \clearfield{venue}
   % \clearfield{volume}
   \clearlist{location} % alias to field 'address'
   \clearlist{publisher}
   \clearname{editor}
}
\addbibresource{../../../biblio/magnet.bib}
% From https://www.writelatex.com/630785fxzchp
% \usepackage[hmargin=2cm,vmargin=2.5cm]{geometry}
\usepackage{fancyhdr}
\newcommand{\userName}{Géraud Le Falher}
\newcommand{\institution}{Inria Lille}
% \pagestyle{fancy}
\rhead{\workingDate}
\chead{\textsc{Research Diary}}
\lhead{\textsc{\userName}}
\rfoot{\textsc{\thepage}}
\cfoot{\textit{Last modified: \today}}
\lfoot{\textsc{\institution}}

\usepackage{wrapfig}
\newcommand{\univlogo}{%
  \noindent % University Logo
  \begin{wrapfigure}{r}{0.3\textwidth}
    \vspace{-2\baselineskip}
    \begin{center}
	    \includegraphics[width=4cm]{../../admin/inria_logo.pdf}
    \end{center}
  \end{wrapfigure}
}

\usepackage{hyperref}
\hypersetup{%
    % draft,    % = no hyperlinking at all (useful in b/w printouts)
    colorlinks=true, linktocpage=true, pdfstartpage=3, pdfstartview=FitV,%
    % uncomment the following line if you want to have black links (e.g., for printing)
    %colorlinks=false, linktocpage=false, pdfborder={0 0 0}, pdfstartpage=3, pdfstartview=FitV,%
    breaklinks=true, pdfpagemode=UseNone, pageanchor=true, pdfpagemode=UseOutlines,%
    plainpages=false, bookmarksnumbered, bookmarksopen=true, bookmarksopenlevel=1,%
    hypertexnames=true, pdfhighlight=/O,%nesting=true,%frenchlinks,%
    urlcolor=Chocolate, linkcolor=DodgerBlue, citecolor=LimeGreen, %pagecolor=RoyalBlue,%
    %urlcolor=Black, linkcolor=Black, citecolor=Black, %pagecolor=Black,%
}

\usepackage[vmargin=2cm]{geometry}
% \addbibresource{../../../biblio/dist.bib}
\newcommand{\stcomp}[1]{\ensuremath{\overline{#1}}}
\newcommand{\allst}{\ensuremath{\mathcal{T}_G}}
\newcommand{\objf}{objective function}
\usepackage{bbm}
\usepackage{setspace}
\usepackage{varioref}
\usepackage{caption}
\usepackage{subcaption}
\title{Partial Low Stretch Trees}
\author{Géraud Le Falher}
\begin{document}
\maketitle

\section*{Problem}

Let $G=(V,E,w)$ be an undirected weighted graph, $X \subset V$ a set of observed
nodes, $\stcomp{X}$ its complement and \allst{} the set of spanning
trees of $G$. We are looking for

\begin{align}
	T^* = \argmin_{T \in \mathcal{T}_G} \; & f(T) \notag \\
	\text{where }\; & f(T) = \sum_{i \in X ,\,j \in \stcomp{X}} \left| path_T(i, j) \right |
	\label{eq:prob}
\end{align}

Note that by symmetry of \eqref{eq:prob}, we may assume that $|X| \leq
\left| \stcomp{X}\right|$. A variant of $f$ is $g$, which measures how closely
a given tree $T$ project $\stcomp{X}$ onto $X$:
\begin{align}
	g(T)= \sum_{j \in \stcomp{X}} \min_{i \in X} \left| path_T(i, j) \right |
\end{align}
Contrary to $f$, $g$ is not symmetric in $X$ and $\stcomp{X}$: consider for
instance the case were $X$ is a singleton.

\section*{Methods}

Let's first consider the unweighted case: $w=\boldsymbol{1}$. Here are three
ideas on how to solve the problem:
\begin{description}

	\item[Merged $k$-BFTs] The first one is quite natural albeit rather
		unpractical: for each $i \in X$, we build an arbitrary Breadth First
		Tree (BFT) rooted at $i$ (let's call it $T_i$. There are several BFTs
		rooted at $i$ hence we could build $k$ of them. In the following
		experiments though, we set $k=1$). This yields the shortest paths from
		$i$ to all $j \in \stcomp{X}$, and we denote the sum of the lengths of
		these $|\stcomp{X}|$ paths as $S_i$. Therefore a (non tight) lower
		bound of $f(T^*)$ is $\sum_{i\in X} S_i$. The last step is then to
		merge these $|X|$ BFT into a single spanning tree. Unfortunately I
		couldn't think of a straightforward way to do that. One method would be
		to count by how many BFT each edges is used, and use the opposite of
		these counts as weights for a minimum spanning tree algorithm, the
		intuition being that we want to preserve edges that are important for
		many nodes in $X$.

	\item[Expanding BFTs] The second is based on the observation that, if $X =
		\{i\}$ is made of a single node, then $T^* = T_i$. Thus if $|X| > 1$,
		we want to extend that property by expanding BFTs simultaneously from
		all $i \in X$. To ensure we get a spanning tree at the end, we need to
		avoid creating cycle. This can be done by maintaining information about
		the connectivity between nodes in $X$. It's motivated by the cycle
		example presented in \autoref{fig:cycle}~\vpageref{fig:cycle}. In that
		case, this method could obtain the optimal solution as the optimal edge
		will be among the last to be reached and thus cut.

		\begin{figure}[hbp]
			\centering
			\includegraphics[width=0.85\linewidth]{cpcycle.pdf}
			\caption{In a simple cycle, drawing a spanning tree $T_e$ amounts
				to cutting one edge $e$. Here $X$ is in orange and the label of
				each edge $e$ is the cost $f(T_e)$ of the associate tree.
				Although I didn't prove it, it makes sense that the
				optimal edge to cut is the one farther away from $X$. \label{fig:cycle}}
		\end{figure}

		In practice, we maintain a forest of tree rooted at each node of $X$.
		We expand them in a Breadth first manner while 
		\emph{\textbf{the following part is unclear; refer to
		\autoref{alg:mbfs} \vpageref{alg:mbfs} instead} More precisely, during the
		construction, each node in $V$ is assigned the index of $i \in X$ from
		which it has been discovered first. When two partial trees are about to
		join, we check whether their respective label are already connected or not.}

	\autoref{fig:cycle}~\vpageref{fig:cycle} also shows that between nodes 0
	and 17, the optimal edge is farther from 0, suggesting than larger subtrees
	(like the one starting at $0-1-2$) should grow faster. Taking into account
	the \textbf{Momentum} of each subtree, upon dequeuing a node, we skip it
	(i.e. enqueue it back) with probability based on the ratio of the size of
	the subtree this node belongs and the size of the currently largest
	subtree.

	\item[Modified SGT] Although BFTs enjoy alluring practical performances,
		they don't provide any theoretical guarantees. Therefore we might adapt
		\emph{Galaxy Tree} construction to that setting, by choosing star
		centers in priority in $X$. Practically, instead of sorting nodes by
		their degree, we sort them in lexicographical order according to $(1 -
		\mathbbm{1}_X, \textrm{degree})$ (that is we put all nodes of $X$
		first). When collapsing the graph, we need to decide how to propagate
		the information about membership to $X$. The simplest way (currently
		implemented in the experiments) is to say that a star (i.e. a node in
		the future collapsed graph) belongs to $X'$ (the successor of $X$ in
		the new graph) if its center belongs to $X$. A more complicated method
		would replace the binary indicator function by a real value
		representing the strength of the tie with $X$ (for instance, a star
		made only of nodes of $X$ should be more strongly associated to $X'$
		than one with only one node in $X$). It's unclear whether this would
		provide any gain.

\end{description}

In addition, we compare with some baselines that don't rely very much on $X$.

\begin{description}
	\item[BFT] A Breadth First tree rooted at (one of) the node with highest
		degree in $X$.
	\item[SGT] The good old short galaxy tree, completely ignoring $X$.
\end{description}

\begin{algorithm}
	\caption{BFT from multiple roots \label{alg:mbfs}}
	\begin{algorithmic}[0]
		\State \textsc{Input:} $G=(V, E), X \subset V$
		\State \textsc{Output:} $T$ a spanning tree of $G$
		\State $T \gets$ an empty tree, $Q \gets$ an empty queue
		\State $\forall i,j \in X, i<j$, let $Conn[i][j] = (i, j) \in E$
		\State $\forall i \in V$, let $label[i] = i \,$if $i\in X$ else $None$
		\ForAll{$x \in X$}
			\State $Q$.enqueue($x$)
		\EndFor
		\While{$Q \neq \emptyset$}
			\Let{$v$}{$Q.$dequeue()}
			\ForAll{$w \in \mathcal{N}(v)$}
				\If{$label[w]$ is $None$}
					\State $Q$.enqueue($w$)
					\Let{$T$}{$T \cup \{(v, w)\}$}
					\Let{$label[w]$}{$label[v]$}
				\ElsIf{$Conn[label[v]][label[w]]$ is true}
					\State continue
				\Else \Comment{$v$ and $w$ are not yet connected through $T$}
					\Let{$T$}{$T \cup \{(v, w)\}$}
					\Let{$Conn[label[v]][label[w]]$}{true}
					\State update $Conn$
				\EndIf
			\EndFor
		\EndWhile
	\end{algorithmic}
\end{algorithm}

\end{document}

\section*{Preliminary Results}

\begin{figure}[htpb]
	\centering
	\begin{subfigure}[b]{0.45\textwidth}
		\includegraphics[width=\textwidth]{cxcycle_35_baselinebft.pdf}
		\caption{Baseline BFT}
	\end{subfigure}~
	\begin{subfigure}[b]{0.45\textwidth}
		\includegraphics[width=\textwidth]{cxcycle_35_baselinesgt.pdf}
		\caption{Baseline SGT}
	\end{subfigure}

	\begin{subfigure}[b]{0.45\textwidth}
		\includegraphics[width=\textwidth]{cxcycle_35_expandingbfts.pdf}
		\caption{Expanding BFTs}
	\end{subfigure}~
	\begin{subfigure}[b]{0.45\textwidth}
		\includegraphics[width=\textwidth]{cxcycle_35_momentume-bfts.pdf}
		\caption{Momentum BFTs}
	\end{subfigure}

	\begin{subfigure}[b]{0.45\textwidth}
		\includegraphics[width=\textwidth]{cxcycle_35_mergedbfts.pdf}
		\caption{Merged BFTS}
	\end{subfigure}~
	\begin{subfigure}[b]{0.45\textwidth}
		\includegraphics[width=\textwidth]{cxcycle_35_modifiedsgt.pdf}
		\caption{Modified SGT}
	\end{subfigure}
	\caption{Results of the different methods and baselines on a simple
		cycle. \label{fig:rcycle}}
\end{figure}


\begin{figure}[htpb]
	\centering
	\begin{subfigure}[b]{0.45\textwidth}
		\includegraphics[width=\textwidth]{cPA3_1024_baselinebft_5_14.pdf}
		\caption{Baseline BFT: $5.14$}
	\end{subfigure}~
	\begin{subfigure}[b]{0.45\textwidth}
		\includegraphics[width=\textwidth]{cPA3_1024_baselinesgt_5_17.pdf}
		\caption{Baseline SGT: $5.17$}
	\end{subfigure}

	\begin{subfigure}[b]{0.45\textwidth}
		\includegraphics[width=\textwidth]{cPA3_1024_expandingbfts_7_88.pdf}
		\caption{Expanding BFTs: $7.88$}
	\end{subfigure}~
	\begin{subfigure}[b]{0.45\textwidth}
		\includegraphics[width=\textwidth]{cPA3_1024_momentume-bfts_7_17.pdf}
		\caption{Momentum BFTs: $7.17$}
	\end{subfigure}

	\begin{subfigure}[b]{0.45\textwidth}
		\includegraphics[height=.3\textheight]{cPA3_1024_mergedbfts_8_46.pdf}
		\caption{Merged BFTS: $8.46$}
	\end{subfigure}~
	\begin{subfigure}[b]{0.45\textwidth}
		\includegraphics[height=.3\textheight]{cPA3_1024_modifiedsgt_6_26.pdf}
		\caption{Modified SGT: $6.26$}
	\end{subfigure}
	\caption{Results of the different methods and baselines on a PA network.
		\label{fig:rpa}}
\end{figure}

\begin{figure}[htpb]
	\centering
	\begin{subfigure}[b]{0.45\textwidth}
		\includegraphics[width=\textwidth]{cGrid_32_baselinebft_35_96.pdf}
		\caption{Baseline BFT: $35.96$}
	\end{subfigure}~
	\begin{subfigure}[b]{0.45\textwidth}
		\includegraphics[width=\textwidth]{cGrid_32_baselinesgt_43_96.pdf}
		\caption{Baseline SGT: $43.96$}
	\end{subfigure}

	\begin{subfigure}[b]{0.45\textwidth}
		\includegraphics[width=\textwidth]{cGrid_32_expandingbfts_51_62.pdf}
		\caption{Expanding BFTs: $51.62$}
	\end{subfigure}~
	\begin{subfigure}[b]{0.45\textwidth}
		\includegraphics[width=\textwidth]{cGrid_32_momentume-bfts_46_93.pdf}
		\caption{Momentum BFTs: $46.93$}
	\end{subfigure}

	\begin{subfigure}[b]{0.45\textwidth}
		\includegraphics[width=\textwidth]{cGrid_32_mergedbfts_52_34.pdf}
		\caption{Merged BFTS: $52.34$}
	\end{subfigure}~
	\begin{subfigure}[b]{0.45\textwidth}
		\includegraphics[width=\textwidth]{cGrid_32_modifiedsgt_52_43.pdf}
		\caption{Modified SGT: $52.43$}
	\end{subfigure}
	\caption{Results of the different methods and baselines on a grid network.
		\label{fig:rgrid}}
\end{figure}
% \begingroup
% \setstretch{0.9}
% \setlength\bibitemsep{2pt}
% \printbibliography
% \endgroup
