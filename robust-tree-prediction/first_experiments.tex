\documentclass[a4paper,final,notitlepage,11pt,svgnames]{article}
\usepackage[english]{babel}
\usepackage[utf8]{inputenc}
\usepackage{amsmath}
\usepackage{amssymb}
\usepackage{microtype}
\usepackage{csquotes}
\usepackage{fullpage}
\usepackage{charter}
\usepackage{xcolor}
\usepackage{enumitem}
\usepackage{setspace}
\usepackage{booktabs}
\usepackage[np,autolanguage]{numprint}
% \usepackage{geometry}
\usepackage[final]{pdfpages}
\usepackage[citestyle=numeric-comp,bibstyle=ieee,isbn=false,maxnames=1,minnames=1,sorting=none,backend=biber,defernumbers=true]{biblatex}
\AtEveryBibitem{
   \clearfield{arxivId}
   % \clearfield{booktitle}
   % \clearfield{doi}
   \clearfield{eprint}
   \clearfield{eventdate}
   \clearfield{isbn}
   \clearfield{issn}
   % \clearfield{journaltitle}
   \clearfield{month}
   % \clearfield{number}
   % \clearfield{pages}
   \clearfield{series}
   % \clearfield{url}
   \clearfield{urldate}
   \clearfield{venue}
   % \clearfield{volume}
   \clearlist{location} % alias to field 'address'
   \clearlist{publisher}
   \clearname{editor}
}
\usepackage{hyperref}
\hypersetup{%
    % draft,    % = no hyperlinking at all (useful in b/w printouts)
    colorlinks=true, linktocpage=true, pdfstartpage=3, pdfstartview=FitV,%
    % uncomment the following line if you want to have black links (e.g., for printing)
    %colorlinks=false, linktocpage=false, pdfborder={0 0 0}, pdfstartpage=3, pdfstartview=FitV,%
    breaklinks=true, pdfpagemode=UseNone, pageanchor=true, pdfpagemode=UseOutlines,%
    plainpages=false, bookmarksnumbered, bookmarksopen=true, bookmarksopenlevel=1,%
    hypertexnames=true, pdfhighlight=/O,%nesting=true,%frenchlinks,%
    urlcolor=Chocolate, linkcolor=DodgerBlue, citecolor=LimeGreen, %pagecolor=RoyalBlue,%
}
\begin{filecontents}{jobname.bib}
@article{WTA13,
  author  = {Nicol{\`o} Cesa-Bianchi and Claudio Gentile and Fabio Vitale and Giovanni Zappella},
  title   = {Random Spanning Trees and the Prediction of Weighted Graphs},
  journal = {Journal of Machine Learning Research},
  year    = {2013},
  volume  = {14},
  pages   = {1251-1284},
  url     = {http://jmlr.org/papers/v14/cesa-bianchi13a.html}
}
@article{ImdbData07,
  author  = {Sofus A. Macskassy and Foster Provost},
  title   = {Classification in Networked Data: A Toolkit and a Univariate Case Study},
  journal = {Journal of Machine Learning Research},
  year    = {2007},
  volume  = {8},
  pages   = {935-983},
  url     = {http://www.jmlr.org/papers/v8/macskassy07a.html}
}

\end{filecontents}
\addbibresource{jobname.bib}

\title{\rta{} first experiments}
\author{Géraud Le Falher}
% \date{}

\newcommand{\rta}{\textsc{RTA}}
\newcommand{\shazoo}{\textsc{Shazoo}}
\newcommand{\lprop}{\textsc{Lab. Prop}}
\newcommand{\cora}{\textsc{Cora}}
\newcommand{\citeseer}{\textsc{Citeseer}}
\newcommand{\pubmed}{\textsc{Pubmed}}
\newcommand{\usps}{\textsc{USPS-50}}
\newcommand{\rcv1}{\textsc{RCV1}}
\newcommand{\imdb}{\textsc{IMDB}}
\begin{document}
\maketitle

\section*{Datasets}
\label{sec:Datasets}

The first 3 datasets under consideration are citations
networks\footnote{Available at
\url{http://linqs.cs.umd.edu/projects/projects/lbc/}} (with directions removed).
Nodes are papers, which belong to a few categories (Agents, AI, DB, IR, ML and
HCI for \citeseer{}; Case Based, Genetic Algorithms, Neural Networks,
Probabilistic Methods, Reinforcement Learning, Rule Learning and Theory for
\cora{}; and \enquote{Diabetes Mellitus, Experimental}, \enquote{Diabetes
Mellitus Type 1}, and \enquote{Diabetes Mellitus Type 2} for \pubmed{}). Edges
between papers are weighted by the cosine similarity between their abstracts
seen as bag of words. I consider only the largest connected component of each
graph, and for \pubmed{}, I further restrict it to its 3-core\footnote{The main
3-core is the largest set of nodes that have at least 3 neighbors in that same
set.}. Then, I built a 50 nearest neighbors graph from 4500 images of digit
chosen at random in the USPS dataset, following the experimental setting
described in~\autocite{WTA13}. The size of those graphs is displayed
in~\autoref{tab:datasets}.

\imdb{}~\autocite{ImdbData07} nodes are movies released in the United States
between 1996 and 2001, they are linked if they share a production company, and
the weight of an edge is the number of such shared production companies. Movies
belong to two classes, depending of whether they grossed more than 2 million
dollars on their opening
weekend\footnote{\url{http://netkit-srl.sourceforge.net/data.html}}.

\begin{table}[hb]
  \centering
  \caption{Datasets statistics}
  \label{tab:datasets}
  \begin{tabular}{lrr}
    \toprule
    Dataset     & $|V|$     & $|E|$      \\
    \midrule
    \imdb{}     & \np{1126} & \np{20282} \\
    \citeseer{} & \np{2110} & \np{3668}  \\
    \cora{}     & \np{2484} & \np{5068}  \\
    \usps{}     & \np{4500} & \np{33121} \\
    \rcv1{}     & \np{4500} & \np{32715} \\
    \pubmed{}   & \np{4201} & \np{21042} \\
    \bottomrule
  \end{tabular}
\end{table}

Later I would like to add the same Reuters Corpus Volume 1 dataset as used
in~\autocite{WTA13}, and another one based on airport traffic. Basically, from
the list of all flights which departed from or arrived at a US airport in
2016, I would build a graph where nodes are airports, connected by an edge
weighted by the number of flights between those two airports. Then I would
divide airports between regional and worldwide ones (based on the data), giving
a binary label the to the nodes. My assumption is that most regional airports are
connected to others regional ones (the same being true for international ones),
meaning there should be a small number of $\Phi$ edges.

\section*{Setup}
\label{sec:Protocol}

I performed the following experiment. For each dataset, I chose 5 sizes of
training set: $2.5\%$, $5\%$, $10\%$, $20\%$ and $40\%$. For each size of
training set, I also chose 5 perturbation levels $p_0$: $0\%$ (that is, no
perturbation at all), $2.5\%$, $5\%$, $10\%$ and $20\%$. In the results section,
the yellow line correspond to the number of $\Phi$ edges (without taking weights
into account) and as expected, it increases with $p_0$.

When we say that the perturbation level is $p_0$ (for instance $p_0=5\%)$, the
actual probability $p_i$ of each node $i$ to have its sign flipped depends of
its degree $d_i$ in the following way. Denoting $\bar{d}$ the average degree and
$\Delta$ the maximum degree, we first map linearly degrees in the interval $[0,
\bar{d}]$ to probability in $[0, p_0]$ and degrees in the interval $[\bar{d},
\Delta]$ to probability in $[p_0, 2p_0]$. Then we rescale the resulting
probabilities so that their average is $p_0$.

Once the training set has been fixed and the labelling perturbed, we repeat the
following procedure 9 times:\footnote{Unfortunately, this does let us know
which part of the variance is due to the choice of the training set and which
part is due to the perturbation (but based on the $0\%$ results, I would say the
training set has only a small influence on the overall variance).}

We draw 17 trees in a breadth first manner from the highest degree node, but
with children visited in a random order. For each of these trees, the training
set is presented with 11 different orders during the online phase, which
introduce a variance in the guilt coefficient computed for nodes of the training
set\footnote{Because of the way the code is organized, it's easier to do the
same for \shazoo{}, although in that case, it has naturally no effect on the
results.}. We predict the remaining nodes in a batch fashion. We use majority
vote on each node to aggregate these $17\times 11 = 187$ predictions into a
single one. Finally we evaluate the performance by the number of times the
predicted label of a node differs from the original (unperturbed label) of that
same node (more shortly, the number of mistakes).

So far the only competitor is label propagation\footnote{\lprop{} implementation
is very fast, as it only requires a constant number of multiplications between
the sparse weighted adjacency matrix of the graph and the binary vector of
labels.} but later I plan to include WTA~\autocite{WTA13} (as soon as I
implement it).

\section*{Results}
\label{sec:Results}

One immediate comment on the results showed in the following pages is that the
number of mistakes is not monotonically increasing with the perturbation level.
First, this does not happen when I compute the mistakes with respect to the
perturbed labelling (i.e. the labelling actually showed to the learner) and
second, I suspect this is an artefact due to the number of repetitions ($9$) not
being large enough. Indeed, if we take for instance the case of the \citeseer{}
dataset with a $5\%$ training set and perturbation going from $2.5\%$ to $5\%$,
and repeat the same experience 60 times, we get the following results:
\begin{center}
  \begin{tabular}{lcc}
    \toprule
    Perturbation & $2.5\%$ & $5\%$ \\
    \midrule
    17*\rta{}+BFS & $304.2 \pm 51$& $311.0 \pm 39$\\
    17*\shazoo{}+BFS & $306.6 \pm 46$& $316.3 \pm 46$\\
    \bottomrule
  \end{tabular}
\end{center}
which shows that the difference in number of mistakes between the two levels of
perturbation is well within one standard deviation, making comparison between few
measurements misleading.
\medskip

Another disturbing fact is that as the training size increases, the performance
of both \shazoo{} and \rta{} does not to seem to improve (and even decreases
sometimes, for instance on \cora{} between $20\%$ and $40\%$ at the $0\%$
perturbation level).  I have no explanation for that so far, although I'm sure
(by looking at the execution time) that both algorithm indeed used a larger
training set during the experiments.

As for the performance themselves, there are a few general trends. \rta{} and
\shazoo{} are close to each other and there is no clear situation where one is
better than the other. Furthermore, \lprop{} is better, although as the
perturbation level increases, \rta{} and \shazoo{} get more competitive on the
two citation datasets. On the other hand, on \usps{}, \lprop{} remains almost always
unchallenged, especially as the training size increases.  Because each nodes has
50 neighbors (and we expect them to be more or less of the same class), a train
set of size $40\%$ is enough to compensate for the perturbation of $20\%$ of the
signs, yet \shazoo{} and \rta{} do not appear to take advantage of that fact.

\newpage
% \newgeometry{hmargin=1.2cm,top=1.2cm,bottom=1cm}
\pagestyle{empty}
\includepdf[pages=-]{res_shazoo_citeseer.pdf}
\includepdf[pages=-]{res_shazoo_cora.pdf}
\includepdf[pages=-]{res_shazoo_usps.pdf}
\includepdf[pages=-]{res_shazoo_pubmed.pdf}
% \restoregeometry

\begingroup
\setstretch{0.9}
\setlength\bibitemsep{2pt}
\printbibliography
\endgroup
\end{document}
